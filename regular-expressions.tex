\documentclass{article}

\usepackage{geometry}
\geometry{
 a4paper,
 total={170mm,257mm},
 left=20mm,
 top=20mm,
}

\begin{document}

\begin{enumerate}
\item $L^+$ contains $\epsilon \iff \epsilon \in L $
\begin{enumerate}
\item Suppose $\epsilon \in L$. Then, since $\epsilon$ is a non-empty finite
concatenation of a word in $L$, $epsilon \in L^+$.
\item Suppose $L^+$ contains $\epsilon$. This means there is some concatentation
$ u_1 \cdot u_2 \cdot ... \cdot u_n = \epsilon$ where each $u_i \in L$
\item We know $a \cdot b = \epsilon \Rightarrow a = \epsilon = b$
\item Hence, $u_1 = u_2 = ... u_n = \epsilon \Rightarrow \epsilon \in L$
\end{enumerate}
\item $L^+$ contains non empty string $w \iff w \in L^*$
\item $L^* = L^+ \cup \{\epsilon\}$
\item $ \epsilon \in L \iff L^+ = L^* $
\begin{enumerate}
\item Suppose $L^* = L^+$ Since $ \epsilon \in L^*$ (by definition), $\epsilon \in L^+$
\item Then $ \epsilon \in L^+ \Rightarrow \epsilon \in L$ (proved previously)
\item Suppose $\epsilon \in L$
\item Then, $\epsilon \in L^+$ (proved previously)
\item Take any $w \in L^*$. If $w$ is non empty, $w \in L^+$ as shown
previously. If $w = \epsilon$, then we already showed $\epsilon \in L^+$ in
statement (d). Hence $ L^* \subseteq L^+$
\item Take any $w \in L^+$. If $w = \epsilon$, then $w \in L^*$ by definition of
$L^*$. If $w$ is non empty, then $w \in L^*$ by definition. Hence, $L^+
\subseteq L^*$
\item Hence, $L^+ = L^*$
\item In fact, we have shown that $\epsilon \in L \iff \epsilon \in L^+ \iff L^+
= L^*$
\end{enumerate}
\end{enumerate}

\end{document}

