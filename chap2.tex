\documentclass{article}
\begin{document}

\begin{enumerate}
\item Find a binary op that is associative but not commutative
\begin{enumerate}
\item Let $ f: Z \times Z \Rightarrow Z $ be given by  $ f(x, y) = y \in Z $
\item Then $ f(f(a, b), c) = f(b, c) = c = f(a, c) = f(a, f(b, c)) $
\item Thus, f is associative.
\item However, $ f(a, b) = b \neq a = f(b, a) $

\end{enumerate}

\item It followss from group axioms that identity element in the group is
unique.
\item It follows from group axioms that for each element, its inverse is unique.
\item Cancellation law follows from group axioms.

\item This binary operation is the concatenation operation for strings.
\begin{enumerate}
\item Let A be the set containing all strings made from the English alphabet.

\item Define $ f: A \times A \Rightarrow A $ by $ f(x, y) = xy $ for $ x, y \in A $

\item For example, f(hello, world) = helloworld
\item We see that $ f(x, y) = xy \neq yx = f(y, x) $
\item Thus, concatentation is not commutative.

\item However, $f(x, f(y, z)) = f(x, yz) = xyz = f(xy, z) = f(f(x, y), z) $

\item Thus, f is associative.

\end{enumerate}

\item Suppose $(G, \ast) $ and $(H, \star)$ are isomorphism groups. If $(G, \ast)$ is abelian, $(H, \star) $ is abelian

\item There exists bijection f from G to H, such that $ f(a \ast b) = f(a)\star f(b) $
\item Take any arbitray x, y in H.
\item Since f is bijective, there exists a, b in G such that $ f(a) = x, f(b) = y $
\item Since G is abelian, $ f(a \ast b) = f(b \ast a) $
\item Combining the above properties, $ x \star y = f(a) \star f(b) = f(a \ast b) = f(b \ast a) = f(a) \star f(b) = y \star x $

\end{enumerate}

\end{document}

