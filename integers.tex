\documentclass{article}
\begin{document}

\textbf{Divisibility facts}
\begin{enumerate}
\item Prove that for any  integers $ a, b, ab = 0 \Rightarrow a = 0 \lor b = 0 $

\begin{enumerate}
\item Suppose $ ab = 0 $ for some integers a, b
\item Suppose $ a \neq 0. $ Then, there exists an integer $ a^{-1} $ such that $aa^{-1} = 1 $
\item Then, $a^{-1} a b = b = 0 $
\item Similarly, $ b \neq 0 \Rightarrow a = 0 $. Proved.
\end{enumerate}

\item Prove that if x,y are positive, then $ x \mid y \Rightarrow x \leq y $
\begin{enumerate}
\item By definition, there exists integer m such that mx = y
\item Since y is positive, m and x are either both negative or both positive.
\item Since x is positive, m must be positive
\item $ m \geq 1 $
\item Multiplying by x on both sides, $ mx \geq x $
\item Since mx = y, $ mx = y \geq x $
\end{enumerate}

\item Prove that if x, y are negative, then $ x \mid y \Rightarrow y \leq x $

\item If n divides 1, $ n = \pm 1 $
\begin{enumerate}
\item Since n divides 1, xn = 1 for some integer x
\item Either both x and n are positive, or both are negative.
\item Suppose both are positive. Then, $ 0 \leq n $ and by previous result, $ n \leq 1 $
\item Thus, n must be 1
\item Suppose both x and n are negative. Then, $ (-x)(-n) = 1 $. By preceding
deduction, -n = 1
\item This means n = -1
\item Thus, $ n = \pm 1 $
\end{enumerate}

\item For any nonzero integer x, if d divides x, then $ -|x| \leq d \leq |x| $
\begin{enumerate}
\item Suppose x is positive.
\item If d is positive, then $ -|x| < 0 \leq d \leq |x| $
\\ Hence, $ -|x| \leq d \leq |x| $
\item If d is negative, then -d is positive and also divides x, implying that $ -|x| \leq -d \leq |x| $
\\ Hence, $ -|x| \leq d \leq |x| $
\item Suppose x is negative.
\item If d is negative, then $ x = -|x| \leq d < 0 < |x| $
\item Hence, $ -|x| leq d \leq |x| $
\item If d is positive, then -d is negative and also divides x, implying that $
-|x| \leq -d \leq |x| $
\\ Hence, $ -|x| \leq d \leq |x| $
\end{enumerate}

\item For any nonzero integer x, integer d divides x iff -d divides x
\begin{enumerate}
\item Suppose d divides x. i.e. x = md, for some integer m
\item Then, x = (-d)*(-m). Thus, -d divides x
\item The same logic can be applied in the converse direction.
\end{enumerate}

\item For any nonzero integer x, its set of divisors is finite.
\item For the integer 0, the set of divisors is countably infinite.

\item Every non zero integer x has at least 1 positive divisor: 1
\end{enumerate}

\textbf{Greatest common divisor facts}

\begin{enumerate}
\item For any 2 integers x and y, their gcd is a positive integer
\begin{enumerate}
\item The proof stems from fact that any integer has at least 1 positive divisor (namely, the integer 1).
\item Hence, if there are no divisors > 1, 1 will be the greatest common divisor
\end{enumerate}
\end{enumerate}

\end{document}
